\chapter{Vectors}

\section{Vector spaces}

\begin{defn}
Let $ F $ be a set and let $ + $ and $ \cdot $ be two operations defined for elements of $ F $. $ F $, together with the operations $ + $ and $ \cdot $, is called a \defnem{field} if all of the following axioms are satisfied:
\begin{enumerate}
    \item $ + $ and $ \cdot $ are associative, i.e. for all $ a,b,c\in F $, we have
    \begin{equation*}
        (a+b)+c=a+(b+c) \quad\text{and}\quad (a\cdot b)\cdot c=a\cdot(b\cdot c).
    \end{equation*}

    \item $ + $ and $ \cdot $ are commutative, i.e. for all $ a,b\in F $, we have
    \begin{equation*}
        a+b=b+a \quad\text{and}\quad a\cdot b=b\cdot a.
    \end{equation*}

    \item There exists an element $ 0_F\in F $, called the \defnem{additive identity}, such that for all $ a\in F $, we have
    \begin{equation*}
        a+0_F=a.
    \end{equation*}

    \item There exists an element $ 1_F\in F $, called the \defnem{multiplicative identity}, such that for all $ a\in F $, we have
    \begin{equation*}
        a\cdot 1_F=a.
    \end{equation*}

    \item For every $ a\in F $, there exists an element $ -a\in F $, called the \defnem{additive inverse} of $ a $, such that
    \begin{equation*}
        a+(-a)=0_F.
    \end{equation*}

    \item For every $ a\in F $ other than $ 0_F $, there exists an element $ a^{-1}\in F $, called the \defnem{multiplicative inverse} of $ a $, such that
    \begin{equation*}
        a\cdot a^{-1}=1_F.
    \end{equation*}

    \item $ \cdot $ is distributive over $ + $, i.e. for all $ a,b,c\in F $, we have
    \begin{equation*}
        a\cdot(b+c)=(a\cdot b)+(a\cdot c).
    \end{equation*}
\end{enumerate}
\end{defn}

\begin{exmp}
Show that the set of real numbers $ \mathbb{R} $, together with standard addition and multiplication, is a field.
\end{exmp}
\begin{sltn}
We will prove this result by examining each of the axioms one-by-one:
\begin{enumerate}
    \item We already know that standard addition and multiplication are associative.

    \item We also know that standard addition and multiplication are commutative.

    \item The additive identity is the number 0.
    
    \item The multiplicative identity is the number 1.
    
    \item For any $ x\in\mathbb{R} $, the additive inverse is the number $ -x $.
    
    \item For any $ x\in\mathbb{R} $ other than 0, the multiplicative inverse is the number $ 1/x $.
    
    \item We already know that standard multiplication is distributive over standard addition.
\end{enumerate}
Hence, $ \mathbb{R} $ is a field.
\end{sltn}

\begin{exmp}
Show that the set of integers $ \mathbb{Z} $, together with standard addition and multiplication, is not a field.
\end{exmp}
\begin{sltn}
The multiplicative identity is the number 1. Consider the number $ 2\in\mathbb{Z} $. There does not exist a number $ n\in\mathbb{Z} $ such that $ 2n=1 $; that is, 2 does not have a multiplicative inverse in $ \mathbb{Z} $. Hence, $ \mathbb{Z} $ is not a field.
\end{sltn}

\begin{defn}
Let $ F $ be a field and $ V $ be a set, and let $ \odot:F\times V\to V $ and $ \oplus:V\times V\to V $ be two binary operations. $ V $, together with these operations, is called a \defnem{vector space} over $ F $ if all of the following axioms are satisfied:
\begin{enumerate}
    \item $ \oplus $ is associative, i.e. for all $ \vect{u},\vect{v},\vect{w}\in V $, we have
    \begin{equation*}
        (\vect{u}\oplus\vect{v})\oplus\vect{w}=\vect{u}\oplus(\vect{v}\oplus\vect{w}).
    \end{equation*}

    \item $ \oplus $ is commutative, i.e. for all $ \vect{u},\vect{v}\in V $, we have
    \begin{equation*}
        \vect{u}\oplus\vect{v}=\vect{v}\oplus\vect{u}.
    \end{equation*}

    \item There exists an element $ \vect{0}\in V $, called the \defnem{zero vector}, such that for all $ \vect{v}\in V $, we have
    \begin{equation*}
        \vect{v}\oplus\vect{0}=\vect{v}.
    \end{equation*}

    \item For every $ \vect{v}\in V $, there exists an element $ -\vect{v}\in V $, called the \defnem{additive inverse} of $ \vect{v} $, such that
    \begin{equation*}
        \vect{v}\oplus(-\vect{v})=\vect{0}.
    \end{equation*}

    \item For all $ a,b\in F $ and $ \vect{v}\in V $, we have
    \begin{equation*}
        a\odot (b\odot\vect{v})=(ab)\odot\vect{v}.
    \end{equation*}

    \item For every $ \vect{v}\in V $, we have
    \begin{equation*}
        1_F\odot\vect{v}=\vect{v}
    \end{equation*}
    where $ 1_F $ is the multiplicative identity of $ F $.

    \item $ \odot $ is distributive over $ \oplus $, i.e. for all $ a\in F $ and $ \vect{u},\vect{v}\in V $, we have
    \begin{equation*}
        a\odot(\vect{u}\oplus\vect{v})=(a\odot\vect{u})\oplus(a\odot\vect{v}).
    \end{equation*}

    \item For all $ a,b\in F $ and $ \vect{v}\in V $, we have
    \begin{equation*}
        (a+b)\odot\vect{v}=(a\odot\vect{v})\oplus(b\odot\vect{v}).
    \end{equation*}
\end{enumerate}

The elements of $ F $ are called \defnem{scalars} and the elements of $ V $ are called \defnem{vectors}. The operation $ \odot $ is called \defnem{scalar multiplication} and the operation $ \oplus $ is called \defnem{vector addition}.
\end{defn}