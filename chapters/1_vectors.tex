\chapter{Vectors}\label{chap:vectors}

\section{Vector spaces}

\begin{defn}
Let $ F $ be a set and let $ + $ and $ \cdot $ be two binary operations on $ F $. $ F $, together with the operations $ + $ and $ \cdot $, is called a \defnem{field} if all of the following axioms are satisfied:
\begin{enumerate}
    \item $ + $ and $ \cdot $ are associative, i.e. for all $ a,b,c\in F $, we have
    \begin{equation*}
        (a+b)+c=a+(b+c) \quad\text{and}\quad (a\cdot b)\cdot c=a\cdot(b\cdot c).
    \end{equation*}

    \item $ + $ and $ \cdot $ are commutative, i.e. for all $ a,b\in F $, we have
    \begin{equation*}
        a+b=b+a \quad\text{and}\quad a\cdot b=b\cdot a.
    \end{equation*}

    \item There exists an element $ 0_F\in F $, called the \defnem{additive identity}, such that for all $ a\in F $, we have
    \begin{equation*}
        a+0_F=a.
    \end{equation*}

    \item There exists an element $ 1_F\in F $, called the \defnem{multiplicative identity}, such that for all $ a\in F $, we have
    \begin{equation*}
        a\cdot 1_F=a.
    \end{equation*}

    \item For every $ a\in F $, there exists an element $ -a\in F $, called the \defnem{additive inverse} of $ a $, such that
    \begin{equation*}
        a+(-a)=0_F.
    \end{equation*}

    \item For every $ a\in F $ other than $ 0_F $, there exists an element $ a^{-1}\in F $, called the \defnem{multiplicative inverse} of $ a $, such that
    \begin{equation*}
        a\cdot a^{-1}=1_F.
    \end{equation*}

    \item $ \cdot $ is distributive over $ + $, i.e. for all $ a,b,c\in F $, we have
    \begin{equation*}
        a\cdot(b+c)=(a\cdot b)+(a\cdot c).
    \end{equation*}
\end{enumerate}

The operation $ + $ is called \defnem{addition}, and the operation $ \cdot $ is called \defnem{multiplication}. For multiplication, we will often use the notation $ a\cdot b=ab $.
\end{defn}

Note that by the definition of the operations $ + $ and $ \cdot $, when we add or multiply two elements of a field, the result must also be an element of the field, i.e. for all $ a,b\in F $ where $ F $ is a field, we must have
\begin{equation*}
a+b\in F \quad\text{and}\quad a\cdot b\in F.
\end{equation*}
This is called \defnem{closure}, and we say that a field must be \defnem{closed} under $ + $ and $ \cdot $.

Let us examine some examples of what is and isn't a field.

\begin{exmp}
Show that the set of real numbers $ \mathbb{R} $, together with standard addition and multiplication, is a field.
\end{exmp}
\begin{sltn}
Much of this proof will involve results that we already know and don't need to show in detail. First, note that since the sum and product of two real numbers is always a real number, $ \mathbb{R} $ is closed under standard addition and multiplication. Now we can check the field axioms:
\begin{enumerate}
    \item Associativity: We already know that standard addition and multiplication are associative.

    \item Commutativity: We also know that standard addition and multiplication are commutative.

    \item Existence of additive identity: The additive identity is the number 0.
    
    \item Existence of multiplicative identity: The multiplicative identity is the number 1.
    
    \item Existence of additive inverse: For any $ x\in\mathbb{R} $, the additive inverse is the number $ -x $.
    
    \item Existence of multiplicative inverse: For any $ x\in\mathbb{R} $ other than 0, the multiplicative inverse is the number $ 1/x $.
    
    \item Distributivity: We already know that standard multiplication is distributive over standard addition.
\end{enumerate}
Hence, $ \mathbb{R} $ with standard addition and multiplication is a field.
\end{sltn}

\begin{exmp}
Show that the set of integers $ \mathbb{Z} $, together with standard addition and multiplication, is not a field.
\end{exmp}
\begin{sltn}
We need only find one axiom that does not hold. The multiplicative identity is the number 1. Consider the number $ 2\in\mathbb{Z} $. There does not exist a number $ n\in\mathbb{Z} $ such that $ 2n=1 $; that is, 2 does not have a multiplicative inverse in $ \mathbb{Z} $. Hence, $ \mathbb{Z} $ is not a field.
\end{sltn}

For simplicity, when working with fields of numbers, we will from now on assume standard addition and multiplication unless otherwise specified, and we will denote the field simply by its set. For example, the field $ \mathbb{R} $ is assumed to mean $ \mathbb{R} $ together with standard addition and multiplication.

\begin{defn}
Let $ F $ be a field and $ V $ be a set, and let $ \odot:F\times V\to V $ and $ \oplus:V\times V\to V $ be two binary operations. $ V $, together with these operations, is called a \defnem{vector space} over $ F $ if all of the following axioms are satisfied:
\begin{enumerate}
    \item $ \oplus $ is associative, i.e. for all $ \vect{u},\vect{v},\vect{w}\in V $, we have
    \begin{equation*}
        (\vect{u}\oplus\vect{v})\oplus\vect{w}=\vect{u}\oplus(\vect{v}\oplus\vect{w}).
    \end{equation*}

    \item $ \oplus $ is commutative, i.e. for all $ \vect{u},\vect{v}\in V $, we have
    \begin{equation*}
        \vect{u}\oplus\vect{v}=\vect{v}\oplus\vect{u}.
    \end{equation*}

    \item There exists an element $ \vect{0}\in V $, called the \defnem{zero vector}, such that for all $ \vect{v}\in V $, we have
    \begin{equation*}
        \vect{v}\oplus\vect{0}=\vect{v}.
    \end{equation*}

    \item For every $ \vect{v}\in V $, there exists an element $ -\vect{v}\in V $\!, called the additive inverse of $ \vect{v} $, such that
    \begin{equation*}
        \vect{v}\oplus(-\vect{v})=\vect{0}.
    \end{equation*}

    \item For all $ a,b\in F $ and $ \vect{v}\in V $, we have
    \begin{equation*}
        a\odot (b\odot\vect{v})=(ab)\odot\vect{v}.
    \end{equation*}

    \item For every $ \vect{v}\in V $, we have
    \begin{equation*}
        1_F\odot\vect{v}=\vect{v}
    \end{equation*}
    where $ 1_F $ is the multiplicative identity of $ F $.

    \item $ \odot $ is distributive over $ \oplus $, i.e. for all $ a\in F $ and $ \vect{u},\vect{v}\in V $, we have
    \begin{equation*}
        a\odot(\vect{u}\oplus\vect{v})=(a\odot\vect{u})\oplus(a\odot\vect{v}).
    \end{equation*}

    \item For all $ a,b\in F $ and $ \vect{v}\in V $, we have
    \begin{equation*}
        (a+b)\odot\vect{v}=(a\odot\vect{v})\oplus(b\odot\vect{v}).
    \end{equation*}
\end{enumerate}

The elements of $ F $ are called \defnem{scalars} and the elements of $ V $ are called \defnem{vectors}. The operation $ \odot $ is called \defnem{scalar multiplication} and the operation $ \oplus $ is called \defnem{vector addition}.

Note that the zero vector is the same as the additive identity under vector addition.
\end{defn}

Let us now examine some examples of what is and isn't a vector space.

\begin{exmp}
Show that $ \mathbb{R}^2 $, together with the operations $ \oplus $ and $ \odot $ defined such that for all $ c\in\mathbb{R} $ and $ (x_1,y_1),(x_2,y_2)\in\mathbb{R}^2 $, we have
\begin{equation*}
    (x_1,y_1)\oplus(x_2,y_2)=(x_1+x_2,y_1+y_2)
\end{equation*}
and
\begin{equation*}
    c\odot(x,y)=(cx,cy)
\end{equation*}
is a vector space over $ \mathbb{R} $.
\end{exmp}
\begin{sltn}
Let $ (x,y),(x_1,y_1),(x_2,y_2),(x_3,y_3)\in\mathbb{R}^2 $ and let $ a,b\in\mathbb{R} $. First we must verify that $ \mathbb{R}^2 $ is closed under the given operations. Since $ (x_1+x_2),(y_1+y_2)\in\mathbb{R} $, we see that
\begin{equation*}
    (x_1,y_1)\oplus(x_2,y_2)=(x_1+x_2,y_1+y_2)\in\mathbb{R}^2,
\end{equation*}
and similarly, since $ cx,cy\in\mathbb{R} $, we also see that
\begin{equation*}
    c\odot(x,y)=(cx,cy)\in\mathbb{R}^2.
\end{equation*}
Thus, $ \mathbb{R}^2 $ is closed under $ \oplus $ and $ \odot $. Now we can check the vector space axioms:
\begin{enumerate}
    \item Associativity of $ \oplus $:
    \begin{align*}
        ((x_1,y_1)\oplus(x_2,y_2))\oplus(x_3,y_3) &= (x_1+x_2,y_1+y_2)\oplus(x_3,y_3) \\
        &= (x_1+x_2+x_3,y_1+y_2+y_3) \\
        &= (x_1,y_1)\oplus(x_2+x_3,y_2+y_3) \\
        &= (x_1,y_1)\oplus((x_2,y_2)\oplus(x_3,y_3)).
    \end{align*}

    \item Commutativity of $ \oplus $:
    \begin{equation*}
        (x_1,y_1)\oplus(x_2,y_2)=(x_1+x_2,y_1+y_2)=(x_2+x_1,y_2+y_1)=(x_2,y_2)\oplus(x_1,y_1).
    \end{equation*}

    \item Existence of zero vector: Consider $ \vect{0}=(0,0) $. We see
    \begin{equation*}
        (x,y)\oplus\vect{0}=(x,y)\oplus(0,0)=(x+0,y+0)=(x,y),
    \end{equation*}
    so $ \vect{0} $, as we have defined it, is the zero vector.

    \item Existence of additive inverse: Consider $ -(x,y)=(-x,-y) $. We see
    \begin{equation*}
        (x,y)\oplus(-(x,y))=(x,y)\oplus(-x,-y)=(x-x,y-y)=(0,0)=\vect{0},
    \end{equation*}
    so $ -(x,y) $, as we have defined it, is the additive inverse of $ (x,y) $.

    \item Compatibility of $ \odot $ with field multiplication:
    \begin{equation*}
        a\odot(b\odot(x,y))=a\odot(bx,by)=(abx,aby)=(ab)\odot(x,y).
    \end{equation*}

    \item Scalar multiplicative identity: Recall that the number 1 is the multiplicative identity of $ \mathbb{R} $. We see
    \begin{equation*}
        1\odot(x,y)=(1x,1y)=(x,y).
    \end{equation*}

    \item Distributivity of $ \odot $ over $ \oplus $:
    \begin{align*}
        a\odot((x_1,y_1)\oplus(x_2,y_2)) &= a\odot(x_1\!+x_2,y_1+y_2)=(a(x_1\!+x_2),a(y_1+y_2)) \\
        &= (ax_1\!+ax_2,ay_1+ay_2)=(ax_1,ay_1)\oplus(ax_2,ay_2) \\
        &= (a\odot(x_1,y_1))\oplus(a\odot(x_2,y_2)).
    \end{align*}

    \item Distributivity of $ \odot $ over field addition:
    \begin{align*}
        (a+b)\odot(x,y) &= ((a+b)x,(a+b)y)=(ax+bx,ay+by) \\
        &= (ax,ay)\oplus(bx,by)=(a\odot(x,y))\oplus(b\odot(x,y)).
    \end{align*}
\end{enumerate}
Hence, $ \mathbb{R}^2 $ with these operations is a vector space over $ \mathbb{R} $.
\end{sltn}

In these examples and in the definition, we used the symbols $ \odot $ and $ \oplus $ for scalar multiplication and vector addition, respectively, in order to help distinguish these operations from addition and multiplication on the field. From now on, we will use additive notation
\begin{equation*}
    \vect{u}\oplus\vect{v}=\vect{u}+\vect{v}
\end{equation*}
for vector addition and multiplicative notation
\begin{equation*}
    c\odot\vect{v}=c\vect{v}
\end{equation*}
for scalar multiplication, but it is still important to remember the distinction between these operations with vectors and the analogous operations on the field.

\sectionnumberless{Exercises}

\begin{exer}\label{exer:cfield}
Consider the set of complex numbers $ \mathbb{C}=\{a+bi\mid a,b\in\mathbb{R}\} $ where $ i $ is the imaginary number, defined such that $ i^2=-1 $. We naturally have the operations $ + $ and $ \cdot $ defined such that for all $ (a+bi),(c+di)\in\mathbb{C} $, we have
\begin{equation*}
    (a+bi)+(c+di)=a+c+bi+di=(a+c)+(b+d)i
\end{equation*}
and
\begin{align*}
    (a+bi)\cdot(c+di) &= ac+a(di)+(bi)c+(bi)(di)=ac+adi+bci+bdi^2 \\
    &= ac+adi+bci-bd=(ac-bd)+(ad+bc)i.
\end{align*}
Show that $ \mathbb{C} $ with these operations is a field.
\end{exer}