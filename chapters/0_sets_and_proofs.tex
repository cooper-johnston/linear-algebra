\chapter{Sets and Proofs}\label{chap:sets_proofs}

\section{Sets}

We will begin by exploring the concept of a set through what is sometimes called intuitive or naive set theory. There exist more rigorous approaches, axiomatic set theories, but we will not be looking at these; our intuitive treatment of sets will suffice for the purposes of this course.

\begin{definition}~
\begin{enumerate}
    \item A \defnem{set} is a well-defined collection of objects. By \textquote{well-defined} we mean that for any set $ S $, any object is either definitely in $ S $ or definitely not in $ S $.
    \item An object that is in a set is called an \defnem{element} of that set. We write $ x\in S $ to denote that $ x $ is an element of the set $ S $.
    \item The set that does not contain any elements is called the \defnem{empty set}, denoted $ \varnothing $.
    \item The number of elements in a set is called the \defnem{cardinality} of that set. We write $ \lvert S\rvert $ to denote the cardinality of the set $ S $.
\end{enumerate}
\end{definition}

One way to describe a set is by listing its elements. For example, we can define $ A $ to be the set containing the numbers 3, 6, 9, and 12, which we write as
\begin{equation*}
    A=\{3,6,9,12\}.
\end{equation*}
Another way is to give a defining property of its elements. For example, $ A $ is the set of the first four positive multiples of three, or more mathematically, $ A $ is the set of all elements $ 3n $ such that $ n=1,2,3,4 $, which we write as
\begin{equation*}
    A=\{3n\mid n=1,2,3,4\}.
\end{equation*}
The latter notation is often called set-builder notation.

We will denote certain special sets of numbers as follows:
\begin{itemize}
    \item $ \mathbb{N} $ is the set of natural numbers, which we will take to start at 1;
    \item $ \mathbb{Z} $ is the set of integers;
    \item $ \mathbb{Q} $ is the set of rational numbers;
    \item $ \mathbb{Q}^+ $ is the set of positive rational numbers;
    \item $ \mathbb{R} $ is the set of real numbers;
    \item $ \mathbb{R}^+ $ is the set of positive real numbers; and
    \item $ \mathbb{C} $ is the set of complex numbers.
\end{itemize}

It is possible to construct some of these using elements of the others. For example, the set of \emph{ratio}nal numbers $ \mathbb{Q} $ is the set of all numbers that can be expressed as a \emph{ratio} $ p/q $ where $ p $ and $ q $ are integers and $ q\neq 0 $, i.e.
\begin{equation*}
    \mathbb{Q}=\left\{\frac{p}{q}\mid p,q\in\mathbb{Z},q\neq 0\right\}.
\end{equation*}
We can also use them to define other sets of numbers, such as the set of even numbers, which is the set of all numbers $ 2n $ where $ n $ is an integer, i.e. $ \{2n\mid n\in\mathbb{Z}\} $.

\begin{definition}
Let $ A $ and $ B $ be two sets. $ B $ is called a \defnem{subset} of $ A $, denoted $ B\subseteq A $, if every element in $ B $ is also an element in $ A $, i.e. if $ b\in B $, then $ b\in A $. $ B $ is called a \defnem{proper subset} of $ A $, denoted $ B\subset A $, if $ B\subseteq A $ and $ B\neq A $. The sets $ B=\varnothing $ and $ B=A $ are called the \defnem{trivial subsets} of $ A $.
\end{definition}

We can see from this definition that
\begin{equation*}
    \mathbb{N}\subset\mathbb{Z}\subset\mathbb{Q}\subset\mathbb{R}\subset\mathbb{C}.
\end{equation*}

Continuous subsets of the real numbers can be expressed as \defnem{intervals}:
\begin{align*}
    & (a,b)=\{x\in\mathbb{R}\mid a<x<b\} && \text{(\defnem{open interval})}, \\
    & [a,b]=\{x\in\mathbb{R}\mid a\leq x\leq b\} &&\text{(\defnem{closed interval})}, \\
    & [a,b)=\{x\in\mathbb{R}\mid a\leq x<b \}, \\
    & (a,b]=\{x\in\mathbb{R}\mid a<x\leq b\}.
\end{align*}

\begin{definition}
Let $ A_1,A_2,\ldots,A_n $ be non-empty sets. The set
\begin{equation*}
    A_1\times A_2\times \cdots\times A_n=\{(a_1,a_2,\ldots,a_n)\mid a_1\in A_1,a_2\in A_2,\ldots,a_n\in A_n\}
\end{equation*}
is called the \defnem{Cartesian product} of $ A_1,A_2,\ldots,A_n $. If $ A_1=A_2=\cdots=A_n=A $, then we can write
\begin{equation*}
    \underbrace{A\times A\times\cdots\times A}_{n\text{ times}}=A^n.
\end{equation*}
\end{definition}

A common example of a Cartesian product is the set of ordered pairs of real numbers, which can be expressed as
\begin{equation*}
    \mathbb{R}^2=\{(x,y)\mid x,y\in\mathbb{R}\}.
\end{equation*}
Note that the notation for an ordered pair $ (x,y)\in\mathbb{R}^2 $ is the same as the notation for an open interval $ (x,y)\subseteq\mathbb{R} $. This motivates an important rule in mathematics: We must be careful to specify what we mean with our notation if it is not clear from the context!

\begin{definition}
Let $ A $ and $ B $ be two sets. The set
\begin{equation*}
    A\cup B=\{c\mid c\in A\text{ or }c\in B\},
\end{equation*}
i.e. the set containing all elements of $ A $ as well as all the elements of $ B $, is called the \defnem{union} of $ A $ and $ B $. The set
\begin{equation*}
    A\cap B=\{c\mid c\in A\text{ and }c\in B\},
\end{equation*}
i.e. the set containing all the elements that are in both $ A $ and $ B $ at the same time, is called the \defnem{intersection} of $ A $ and $ B $.
\end{definition}

\begin{example}
Let $ A=\{1,2\} $ and $ B=\{2,4,6\} $. Express $ A\times B $, $ A\cup B $, and $ A\cap B $ by listing their elements.
\end{example}
\begin{solution}~
\begin{enumerate}
    \item $ A\times B $ is the set of all ordered pairs where the first element is in $ A $ and the second is in $ B $:
    \begin{equation*}
        A\times B=\{(1,2),(1,4),(1,6),(2,2),(2,4),(2,6)\}.
    \end{equation*}
    \item $ A\cup B $ is the set of all elements that are in either $ A $ or $ B $ (or both):
    \begin{equation*}
        A\cup B=\{1,2,4,6\}.
    \end{equation*}
    \item $ A\cap B $ is the set of all elements that are in both $ A $ and $ B $:
    \begin{equation*}
        A\cap B=\{2\}. \qedhere
    \end{equation*}
\end{enumerate}
\end{solution}

\subsection*{Exercises}

\begin{problem}\label{prb:setops}
For each of the following, find $ A\cup B $, $ A\cap B $, $ A\times B $, and $ B\times A $.
\begin{enumerate}[label=(\alph*)]
    \item $ A=\{-1,1\} $, $ B=\{1,2,3\} $.
    \item $ A=\{0,1\} $, $ B=\{3,9,27\} $.
    \item $ A=[-1,1] $, $ B=(0,\infty) $.
\end{enumerate}
\end{problem}

\begin{problem}\label{prb:Z}
Express the set of integers $ \mathbb{Z} $ in terms of the set of natural numbers $ \mathbb{N} $ and the set $ \{0\} $.
\end{problem}

\begin{problem}\label{prb:C}
Express the set of complex numbers $ \mathbb{C} $ in terms of the set of real numbers $ \mathbb{R} $ and the imaginary number $ i $.
\end{problem}

\section{Mathematical logic}

\section{Mappings}

% \begin{definition}
% Let $ A $ and $ B $ be two non-empty sets, and let $ \mathcal{R}\subseteq A\times B $. The set $ \mathcal{R} $ is called a \defnem{relation} between $ A $ and $ B $. For an ordered pair $ (a,b)\in\mathcal{R} $, we say that $ \mathcal{R} $ relates $ a $ to $ b $.
% \end{definition}

% \begin{definition}
% Let $ A $ and $ B $ be two non-empty sets. A relation $ f $ between $ A $ and $ B $ is called a \defnem{mapping} or a \defnem{function} if for every $ a\in A $, there exists exactly one $ b\in B $ such that $ f $ relates $ a $ to $ b $. The set $ A $ is called the \defnem{domain} of $ f $, and $ B $ is called the \defnem{codomain} of $ f $. We write
% \begin{align*}
%     f:A\to B,\quad & a\mapsto b \\
%     & (\text{or}\ f(a)=b)
% \end{align*}
% to denote that $ f $ is a mapping with domain $ A $ and codomain $ B $ (we say in this case that $ f $ is a mapping \defnem{from $ A $ to $ B $}) and that $ f $ \defnem{maps $ a $ to $ b $}.
% \end{definition}

\begin{definition}
Let $ A $ and $ B $ be two non-empty sets. A subset $ f\subseteq A\times B $ is called a \defnem{mapping} or a \defnem{function} from $ A $ to $ B $ if for all $ a\in A $, there exists exactly one $ b\in B $ such that $ (a,b)\in f $ (when we say that a mapping is \defnem{well-defined}, we are referring to this property).

The set $ A $ is called the \defnem{domain} of $ f $ and the set $ B $ is called the \defnem{codomain}. For any pair $ (a,b)\in f $, we write $ f(a)=b $ or $ a\mapsto b $ and say $ f $ \defnem{maps $ a $ to $ b $}. We write
\begin{align*}
    f:A\to B,\quad & a\mapsto b \\
    & (\text{or}\ f(a)=b)
\end{align*}
to denote that $ f $ is a mapping from $ A $ to $ B $ that maps elements $ a\in A $ to corresponding elements $ b\in B $.
\end{definition}

\begin{definition}
Let $ f:A\to B $ be a mapping. The set $ \{f(a)\mid a\in A\} $ is called the \defnem{range} of $ f $, denoted $ \range(f) $.
\end{definition}

\begin{SCfigure}[0.8][hb]
\centering
\begin{tikzpicture}
    % Elements
    \node (a1) at (-2,0.75) {$ a_1 $};
    \node (a2) at (-2,0) {$ a_2 $};
    \node (a3) at (-2,-0.75) {$ a_3 $};
    \node (b1) at (2,0.75) {$ b_1 $};
    \node (b2) at (2,0) {$ b_2 $};
    \node (b3) at (2,-0.75) {$ b_3 $};

    % Set ellipses
    \node[draw,ellipse,fit= (a1) (a2) (a3),minimum width=2cm,label=above: $ A $] {};
    \node[draw,ellipse,fit= (b1) (b2) (b3),minimum width=2cm,label=above: $ B $] {};
    % \node[draw=cyan,thick,ellipse,fit= (b1) (b2),minimum width=1.5cm,label={[text=cyan]right:range}] {};

    % Arrows
    \draw[->] (a1) -- (b1);
    \draw[->] (a2) -- (b2);
    \draw[->] (a3) -- (b2);
\end{tikzpicture}
\caption{An example of a mapping from $ A=\{a_1,a_2,a_3\} $ to $ B=\{b_1,b_2,b_3\} $}
\label{fig:map1}
\end{SCfigure}

If we have a mapping with a finite domain and codomain, we can represent it with a diagram like in Figure \ref{fig:map1}. We see that the range of the mapping shown in this diagram is $ \{b_1,b_2\} $ since these are the elements in $ B $ that are mapped to.

\begin{definition}
A mapping $ f:A\to B $ is called \defnem{injective} if for all $ a,\tilde{a}\in A $, if $ f(a)=f(\tilde{a}) $, then $ a=\tilde{a} $; in other words, no two different elements in $ A $ map to the same element in $ B $. $ f $ is called \defnem{surjective} if for all $ b\in B $, there exists an $ a\in A $ such that $ f(a)=b $. A mapping that is both injective and surjective is called \defnem{bijective}.
\end{definition}

\begin{SCfigure}[0.8][ht]
\centering
\begin{tikzpicture}
    % Elements
    \node (a1) at (-2,0.75) {$ a_1 $};
    \node (a2) at (-2,0) {$ a_2 $};
    \node (a3) at (-2,-0.75) {$ a_3 $};
    \node (b1) at (2,0.75) {$ b_1 $};
    \node (b2) at (2,0) {$ b_2 $};
    \node (b3) at (2,-0.75) {$ b_3 $};

    % Set ellipses
    \node[draw,ellipse,fit= (a1) (a2) (a3),minimum width=2cm,label=above: $ A $] {};
    \node[draw,ellipse,fit= (b1) (b2) (b3),minimum width=2cm,label=above: $ B $] {};

    % Arrows
    \draw[->] (a1) -- (b1);
    \draw[->] (a2) -- (b2);
    \draw[->] (a3) -- (b3);
\end{tikzpicture}
\caption{An example of a bijective mapping from $ A=\{a_1,a_2,a_3\} $ to $ B=\{b_1,b_2,b_3\} $}
\end{SCfigure}

\begin{theorem}\label{thm:surjective}
A mapping $ f:A\to B $ is surjective if and only if $ \range(f)=B $.
\end{theorem}
\begin{proof}
Exercise \ref{prb:surjective}\noqed
\end{proof}

\begin{definition}
Let $ f:A\to B $ be a bijective mapping. The mapping $ f^{-1}:B\to A $, $ f^{-1}(f(a))=a $ is called the \defnem{inverse mapping} of $ f $.
\end{definition}

\subsection*{Exercises}

\begin{problem}\label{prb:surjective}
Prove Theorem \ref{thm:surjective}.
\end{problem}