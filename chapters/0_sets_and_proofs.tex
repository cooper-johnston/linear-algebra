\chapter{Sets and Proofs}

\section{Sets}

We will begin by exploring the concept of a set through what is sometimes called intuitive or naive set theory. This intuitive treatment of sets will suffice for the purposes of this course. A more rigorous approach, axiomatic set theory, is outside the scope of this course.

\begin{defn}
A \defnem{set} is a well-defined collection of objects. By \textquote{well-defined} we mean that for any set $ S $, any object is either definitely in $ S $ or definitely not in $ S $.

An object that is in a set is called an \defnem{element} of that set. We write $ x\in S $ to denote that $ x $ is an element of the set $ S $.

The set that does not contain any elements is called the \defnem{empty set}, denoted $ \varnothing $.

The number of elements in a set is called the \defnem{cardinality} of that set. We write $ \lvert S\rvert $ to denote the cardinality of the set $ S $.
\end{defn}

One way to describe a set is by listing its elements. For example, we can define $ A $ to be the set containing the numbers 3, 6, 9, and 12, denoted by
\begin{equation*}
    A=\{3,6,9,12\}.
\end{equation*}
Another way is to give a defining property of its elements. For example, $ A $ is the set of the first four positive multiples of three, or more mathematically, $ A $ is the set of all elements $ 3n $ such that $ n=1,2,3,4 $, denoted by
\begin{equation*}
    A=\{3n\mid n=1,2,3,4\}.
\end{equation*}
The latter notation is often called set-builder notation.

We will denote certain special sets of numbers as follows:
\begin{align*}
    &\mathbb{Z} \text{ is the set of integers;} \\
    &\mathbb{Z}^+ \text{ is the set of positive integers;} \\
    &\mathbb{Q} \text{ is the set of rational numbers;} \\
    &\mathbb{Q}^+ \text{ is the set of positive rational numbers;} \\
    &\mathbb{R} \text{ is the set of real numbers;} \\
    &\mathbb{R}^+ \text{ is the set of positive real numbers; and} \\
    &\mathbb{C} \text{ is the set of complex numbers.}
\end{align*}

\begin{exmp}
The set of even numbers is the set of all numbers $ 2n $ where $ n $ is an integer, i.e. $ \{2n\mid n\in\mathbb{Z}\} $.
\end{exmp}

\begin{exmp}
The set of \textbf{ratio}nal numbers $ \mathbb{Q} $ is the set of all numbers that can be expressed as a \textbf{ratio} $ p/q $ where $ p $ and $ q $ are integers and $ q\neq 0 $, i.e. $ \mathbb{Q}=\{p/q\mid p,q\in\mathbb{Z},q\neq 0\} $.
\end{exmp}

\begin{defn}
Let $ A $ and $ B $ be two sets. $ B $ is called a \defnem{subset} of $ A $, denoted $ B\subseteq A $, if every element in $ B $ is also an element in $ A $, i.e. for every $ b\in B $, we have $ b\in A $.

$ B $ is called a \defnem{proper subset} of $ A $, denoted $ B\subset A $, if $ B\subseteq A $ and $ B\neq A $.
\end{defn}

\begin{defn}
Let $ A_1,A_2,\ldots,A_n $ be non-empty sets. The set
\begin{equation*}
    A_1\times A_2\times \cdots\times A_n=\{(a_1,a_2,\ldots,a_n)\mid a_1\in A_1,a_2\in A_2,\ldots,a_n\in A_n\}
\end{equation*}
is called the \defnem{Cartesian product} of $ A_1,A_2,\ldots,A_n $.
\end{defn}

The Cartesian product of a set with itself can be denoted by
\begin{equation*}
    \underbrace{A\times A\times\cdots\times A}_{n\text{ times}}=A^n.
\end{equation*}
For example, $ \mathbb{R}\times\mathbb{R}=\mathbb{R}^2 $, the set of ordered pairs of real numbers.

\begin{exmp}
Let $ A=\{1,2\} $ and $ B=\{1,2,3\} $. Then,
\begin{align*}
    A\times B &= \{(a,b)\mid a\in A,b\in B\} \\
    &= \{(1,1),(1,2),(1,3),(2,1),(2,2),(2,3)\}.
\end{align*}
\end{exmp}

\section{Mappings}

\begin{defn}
Let $ A $ and $ B $ be two non-empty sets, and let $ \mathcal{R}\subseteq A\times B $. The set $ \mathcal{R} $ is called a \defnem{relation} between $ A $ and $ B $. For an ordered pair $ (a,b)\in\mathcal{R} $, we say that $ \mathcal{R} $ relates $ a $ to $ b $.
\end{defn}

\begin{defn}
Let $ A $ and $ B $ be two non-empty sets. A relation $ f $ between $ A $ and $ B $ is called a \defnem{mapping} or a \defnem{function} if for every $ a\in A $, there exists exactly one $ b\in B $ such that $ f $ relates $ a $ to $ b $. The set $ A $ is called the \defnem{domain} of $ f $, and $ B $ is called the \defnem{codomain} of $ f $.

We write $ f:A\to B $ to denote that $ f $ is a mapping with domain $ A $ and codomain $ B $; that is, $ f $ is a mapping from $ A $ to $ B $.

We write $ f(a)=b $ or $ a\mapsto b $ to denote that $ f $ relates $ a $ to $ b $; that is, $ f $ maps $ a $ to $ b $.
\end{defn}

\begin{defn}\label{defn:binop}
Let $ A $ and $ B $ be two sets and let $ \odot:B\times A\to A $. The mapping $ \odot $ is called a \defnem{binary operation}. If $ A=B $, i.e. we have $ \odot:A\times A\to A $, we say $ \odot $ is a binary operation on $ A $.

We write $ x\odot y=z $ to denote that $ \odot $ maps $ (x,y) $ to $ z $.
\end{defn}

\section{Propositional logic}

\section{Proofs}